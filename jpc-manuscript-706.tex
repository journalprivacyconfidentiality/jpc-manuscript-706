\documentclass{jpcfinal} %%% last changed 2014-08-20

% JPC Layouting Macros
% THESE ARE ADDED BY THE EDITORIAL TEAM - NO NEED TO SET HERE
\newcommand{\doisuffix}{683.reminiscences}
% \jpcheading{vol}{issue}{year}{notused}{subm}{publ}{rev}{spec_iss}{title}
\jpcheading{8}{1}{2018}{}{Dec~20, 2018}{Dec 2018}{}{Issue in honor of S. Fienberg}
%%% last changed 2018-06-29 =====================================
\usepackage[sort]{natbib}

%% mandatory lists of keywords
\keywords{editorial, relaunch}

%% read in additional TeX-packages or personal macros here:
%% e.g. \usepackage{tikz}
%\usepackage{hyperref}
\usepackage{natbib}
\usepackage[ruled]{algorithm2e}
\usepackage{pdfpages}
\usepackage{acronym}
\input{jpcboxed.tex}
\input{acrodefs.tex}

\newcommand{\urlcite}[2]{#2\footnote{\url{#1}}}
\newcommand{\urlcitex}[3]{#2\footnote{\href{#1}{#3}}}
\begin{document}

\title[Relaunching JPC]{Relaunching the Journal of Privacy and Confidentiality}

%\author{Cynthia Dwork}	%optional
%\address{Harvard University}	%optional
%\email{dwork@seas.harvard.edu}  %optional
%\thanks{This research was supported by the National Science Foundation grants SES-1131897 and ACI-1443014.}	%optional
\author{Lars Vilhuber}	%optional
\address{Managing Editor, Journal of Privacy and Confidentiality\newline 
Department of Economics, Cornell University}	%optional
\email{lars.vilhuber@cornell.edu}  %optional
%\thanks{Managing Editor, JPC}	%optional

\maketitle

This issue is a new start. \cite{DworkEditorial} outlines how we arrived at this point, and where we are heading. In this note, I will focus on the technical aspects of making the journal work, in the past, right  now, and hopefully for some time to come. 
\section{The Past}

Until early in 2018, the primary home of the Journal was at Carnegie Mellon University (CMU), supported by the Department of Statistics, the university libraries, by  Managing Editor Kira Bokalders, and above all, until his untimely death,  by Stephen E. Fienberg. The submission workflow and the journal itself were hosted on the ``Digital Commons'' platform by \href{https://www.bepress.com}{bepress} (called ``Research Showcase'' at CMU). Steve and Kira, with the support of the CMU Libraries, had carried the journal's operations forward in meticulous, reliable, and transparent fashion. However, with Steve's passing, the lack of a local editor was a matter of concern. In early 2018, the Journal's editorial board made the decision to move the editorial office to Cornell University, and to address some of the structural issues that the journal had encountered. I took on the challenging position of Managing Editor, rolled up my sleeves, and started to research our options. We had been an open-access (OA), online-only journal up to this point, and we emphatically wanted to continue along those lines. Going with one of the big commercial publishers or university presses was unlikely to be acceptable to both sides.  Our constraints were a near-total absence of a (long-term) budget (see below),  the need to be accessible to authors from many different disciplines,%
\footnote{Strangely enough, it seems that   not every discipline is comfortable with \LaTeX{} or Javascript for the writing of articles.}
and the ability to be flexible in terms of hosting options in the near future, given the uncertainty surrounding the journal's structure.
%This section will discuss the technical issues addressed during the move, the next section how the structural issues will be addressed in the coming months and years.

One of the first steps was to identify and adopt a journal-hosting platform. Our first choice was to move to a bepress instance at Cornell University, but the \urlcitex{https://web.archive.org/web/20181220151809/https://www.elsevier.com/about/press-releases/corporate/elsevier-acquires-bepress,-a-leading-service-provider-used-by-academic-institutions-to-showcase-their-research}{acquisition of bepress by Elsevier}{See announcement archived on  https://web.archive.org/.} lead Cornell to reconsider it's longterm plans with bepress. We also considered  \urlcite{http://scholasticahq.com/}{Scholastica's}  platform for both submission workflow and publication among the commercial solutions. Furthermore, we looked at some of the ``overlay journal'' options, such as \textit{Project Muse} and \urlcite{https://www.episciences.org/}{episciences.org}, both of which leverage open archives such as \urlcite{https://arxiv.org}{arxiv.org} for storing the articles themselves, and only add on a layer of peer-review and virtual republication.\footnote{Scholastica also allows for this type of publication.} 
%
However, not all of our disciplines are represented on arxiv.org, making platforms like episciences.org and Scholastica's overlay option less attractive. Inquiries to \textit{Project Muse} suggested that it might not be the ideal home for many of our disciplines, either. We thus stepped back from the overlay model, and considered our options for a self-contained publishing solution. Ultimately, we chose the \urlcite{https://pkp.sfu.ca/ojs/}{Open Journal System (OJS)}  created and maintained by the \urlcite{https://pkp.sfu.ca/}{Public Knowledge Project}, because it allowed to us to be flexible with the hosting platform --- for now, PKP also provides hosting and technical support, but we can move to a self-hosted instance at any time --- and with the type of inputs we accept from the various disciplines. The ability to modify the platform, if necessary, and to ``export'' our data in an open format was important as well. For now, we are quite happy, albeit with some quibbles, which we regularly send to receptive PKP support staff. However, the OA world is evolving rapidly, and other options regularly become available.  

There were challenges with the migration. Our historical archive needed to be carried over as well, to provide a consistent view of the journal's issues to viewers. However, no bepress instance had the same export format, and OJS did not have an importer for bepress packages. However, OJS does have the ability to import a consistently structured XML archive. We ultimately wrote our own scripts to convert an idiosyncratic export of data and metadata (\url{https://github.com/journalprivacyconfidentiality/jpc-migration}). Combined with some manual editing, we cleaned up the metadata somewhat, and successfully imported the back issues.

%Once we had a platform, we had to populate it. 
We carried forward much of the descriptive text from the old CMU-hosted platform. Our goal is to have authors handle much of the layouting of the articles, relying on the decade-old promise of using \TeX~ \citep{Knuth1986} and \LaTeX~ \citep{Lamport1986} for distributed typesetting, thus reducing the need for copy editing. Time will tell if that promise has ever held. We adapted a \LaTeX~ class originally created by the journal ``\urlcite{https://lmcs.episciences.org/}{Logical Methods in Computer Science}'' (many thanks to Lars Birkedal, who gave generously of his time and advice), and posted it (\url{https://github.com/journalprivacyconfidentiality/jpc-style}). We developed a workaround for authors submitting in Word, whereby we simply format a Word document with broad margins, allowing us to generate a PDF file that in turn is imported into a styled \LaTeX~ wrapper. We are monitoring closely how this process works, and will provide an assessment at a later stage. In the future, we may consider offering authors that are not comfortable in \LaTeX~ a paid option for converting from Word to \LaTeX, similar to other journals.\footnote{For instance, our friends at \href{https://www.sociologicalscience.com/}{Sociological Science}, a very successful OA journal in sociology, charge between  ``\$50 and \$200, depending on complexity, with the cost for most manuscripts being under \$100'' \citep{SociologicalScience2018}. }

\begin{wrapfigure}{r}[10pt]{2.2in}
	%\begin{figure}[H]
	\begin{cornerbox}[width=2in]
		%		
		\includegraphics[width=1.5in]{tmpl-images/JPC_4.jpg}
		%		
		%		\hfill \it Jerry Reiter
		\caption{Our new logo}
		
	\end{cornerbox}
	%\end{figure}
\end{wrapfigure}
We also formalized and implemented a few more features of OA journals that were previously absent. We registered an \ac{ISSN} with the Library of Congress (for reference, it is 2575-8527). Furthermore, we registered with \urlcite{https://www.crossref.org/}{Crossref}, and registered \ac{DOI} for all previously published articles (for reference, our \ac{DOI} prefix is 10.29012).  We registered a domain (\href{https://www.journalprivacyconfidentiality.org}{journalprivacyconfidentiality.org}), which, if you are reading this, you have found. We created a logo, thanks to the efforts of our editorial board member, Alan Karr. And after extensive discussions, and research into best practices and options, we selected a formal open-access license under which henceforth the journal would publish articles, re-affirming but modernizing the original stance of the founders that the journal should be open, accessible, and free to read.%
\footnote{Previously, a simple statement regarding liberal re-use of article materials was posted.}
Articles henceforth will be under a  \urlcite{https://creativecommons.org/licenses/by-nc-nd/4.0/}{Commons License – Attribution-NonCommercial-NoDerivatives 4.0 International} (otherwise known as CC BY-NC-ND 4.0), unless authors choose a more lenient license (for instance, public domain). 



With those foundations in place, we migrated the journal to the new platform in February 2018. 
Each of the prior issues now has a mention acknowledging the former hosting at CMU.%
\footnote{
	For instance, the first issue of the journal carries the mention ``This issue was first published at Research Showcase @ CMU (\url{http://repository.cmu.edu/jpc/vol1/iss1/}). Research Showcase @ CMU is Carnegie Mellon University's institutional repository.'' The first issue is now accessible at \url{https://doi.org/10.29012/jpc.v1i1}.}
The old website  carried a note, redirecting to the new website. Starting in July 2018, the old website actively redirected to the new website. The new Journal of Privacy and Confidentiality was ready to go.

\section{Where We Are Now}
Naturally, a pretty website does not a journal make. Content was required, and the editors reached out to their various disciplines, spreading the word that the journal was accepting submissions again. We quickly realized that the first relaunch issue needed to be a special one, and Aleksandra Slavkovi\'c and I set out to put this issue together, in honor of Steve \citep{SesaLarsEditorial}. Our workflow developed, and will surely change in the near future. All of the articles of this issue except one were provided as \LaTeX{} articles, and yet, the OJS system is not as streamlined as it could be for that. For each article, we would receive a PDF, which was sent to review. After acceptance, the individual components of a \LaTeX{} article were submitted by authors to the OJS, until we realized that a ZIP package was much easier (I apologize for the troubles those early authors went through). We downloaded each article, stashed it in a (private) Git repository (on Github.com), and then copy-edited on Overleaf.com (a big thanks to Melissa Colbeth for her excellent copy-editing work). Once the article was ready, the PDF would be downloaded from Overleaf.com, and uploaded into OJS. Clearly, this process could be streamlined, starting with sharing of Overleaf documents or Github repositories between authors and editors, and expanding to  integration of Overleaf, Github, and similar platforms through an application programming interface into OJS. We are working on that. 

\section{The Next Steps}

With this issue, the efforts to migrate and stabilize the journal have born fruit. The future success of the journal hinge, of course, critically on the quality of the contributions of its authors, which in turn are disseminated and read by its readers.

Organizationally, the journal continues to be managed by an editorial board, currently consisting of Editor-in-Chief 
Cynthia Dwork, Editors 
John M. Abowd, 
Kobbi Nissim, 
Alan F. Karr, and 
Managing Editor
Lars Vilhuber. In this editorial, we have re-affirmed our commitment to open access and integrity. As we move forward, we will instantiate that commitment in various more formal ways. First, we are looking into putting the Journal's institutional structure onto more a more persistent foundation. We are gathering  input from editors at other open access journals (in particular our colleagues at \urlcite{https://www.sociologicalscience.com/}{Sociological Science}), and we will be making decisions about such a structure in 2019. We have started the process to trademark the Journal's name and logo.

Part of the considerations are financial, as providing open access is ``free as in beer'' for the reader, but not cheap. 
%This may include addressing, in a broadly acceptable and equitable fashion, the potential for article processing charges, which are common in some disciplines, and completely absent in others. 
Currently, costs are covered through a generous subsidy by John Abowd's 
 \urlcite{https://blogs.cornell.edu/abowd/edmund-ezra-day/}{Edmund Ezra Day Chair}  at Cornell University, for three years. We are discussing with colleagues at other OA journals about the right models (looking with admiration on the great work done at Sociological Sciences, for instance), again taking into account that different disciplines have different approaches to this.
 
Our history of unbiased editors and referees speaks for itself, but does not always speak to new potential authors. We are reviewing our policies, will make explicit policies that have so far been implicit, and will post these on our website over the course of the next 3-6 months. Registration in the \urlcite{https://doaj.org/}{Directory of Open Access Journals} and membership in \urlcite{https://publicationethics.org/}{Committee on Publication Ethics (COPE)} are goals for 2019, and amount to a peer-review of the journal's policies. 

As noted above, we will continue working on streamlining the submission experience for our authors, readers, and editors, improving the user experience for all. Our deadlines for reviews and re-submissions may be revised. We will also be expanding our pool of editors, to allow for faster turnaround times. 

And above all, we will strive to remain a forum for lively, informed, and well-written research on the topics of privacy and confidentiality.

\hfill \it Lars Vilhuber

\bibliography{paper}
\bibliographystyle{abbrvnat}

\end{document}